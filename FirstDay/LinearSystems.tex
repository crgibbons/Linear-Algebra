\documentclass[hidelinks,12pt,handout]{ximera}
\usepackage{amssymb,amsmath,graphicx,hyperref,amsthm}
\usepackage{fancyhdr}
\usepackage{fullpage}

\hypersetup{ pdfnewwindow=true, pdffitwindow=false }

\newcommand{\defi}[1]{\textbf{\textit{#1}}}

\title{Linear Systems}
\begin{document}
\begin{abstract}{What puts the ``linear'' in ``Linear Algebra?'' Turns out that this adjective refers to properties of linear systems.}\end{abstract}
\maketitle

\begin{definition}
\begin{itemize}
\item If a linear system has at least one solution, we say the system is \defi{consistent}.
\item If a linear system has exactly one solution, we say the solution is \defi{unique}.
\item If a linear system has no solutions, we say the system is \defi{inconsistent}.
\item If two linear systems have exactly the same solutions, we call them \defi{equivalent}.
\item If $b_1 = b_2 = \cdots = b_m = 0$ in a linear system with $m$ equations, we say the system is \defi{homogeneous}.
\item If $x_1 = x_2 = \cdots = x_n = 0$ is a solution to a linear system in $n$ unknowns, we say the system has the \defi{trivial solution}.
%\item If for some $j$, $1 \leq j \leq n$, $x_j \not = 0$ and $x_j$ is part of a solution to a linear system in $n$ unknowns, we say the system has a \defi{nontrivial solution}.
\end{itemize}
\end{definition}

For each of the following, write down an example or explain why one can't exist:
\begin{question}
an inconsistent linear system in 2 unknowns;
\vfill
\end{question}

\begin{question}
an inconsistent homogeneous linear system in 2 unknowns;
\vfill
\begin{hint} Try starting with one equation, like $x+y = 0$, and then trying to come up with another equation that, when added, makes the system inconsistent. Can you do it? \end{hint}
\end{question}

\begin{question}a unique solution to the linear system 
	\begin{eqnarray*} x + 4 y & = & 12 \\
				   3x + 8y &=&  4;
	\end{eqnarray*}
	\vfill
\end{question}

\begin{question}
a nonhomogeneous system with the trivial solution;
\vfill
\end{question}

\begin{question} a linear system with three equations that is equivalent to the linear system
	\begin{eqnarray*} x + 4 y & = & 12 \\
				   3x + 8y &=&  4.
	\end{eqnarray*}
	\vfill
\end{question}

\begin{question} Jot down any conjectures you have about the way these various properties are related.

\begin{hint} A \textit{conjecture} is a statement that you think is true but that doesn't have a proof (yet). 
\end{hint}
\vfill
\end{question}
\end{document}


